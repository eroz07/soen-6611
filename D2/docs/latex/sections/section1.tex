\section{Problem 3: Effort Estimation}

\subsection{Effort estimation using Use Case Points (UCP) approach}
We only have one actor (complex one) who interacts with the system using graphical user interface to do use cases, including import data (simple), generate data (simple) and calculate statistics measurements (average).

\begin{enumerate}
\item {Calculating Unadjusted Actor Weight(UAW)}
\begin{align}
    UAW     &= 3
\end{align}

\item {Calculating Unadjusted Use Case Weight (UUCW)}
\vspace{8pt}
\begin{align}
    UUCW     &= 2 * 5 + 10\\
            &= 20
\end{align}

\item {Calculating Technical Complexity Factors (TCF)}

\begin{table}[!thb]
\begin{tabularx}{\textwidth}{|l|X|l|l|}
\hline
\textbf{TCF Type} & \textbf{Factor Name} & \textbf{Weight} & \textbf{Perceived Impact Factor} \\
\hline
T1 & Distributed System                     & 2 & 0 \\\hline
T2 & Performance                            & 1 & 3 \\\hline
T3 & End User Efficiency                    & 1 & 3 \\\hline
T4 & Complex Internal Processing            & 1 & 3 \\\hline
T5 & Reusability                            & 1 & 3 \\\hline
T6 & Easy to Install                        & 0.5 & 5\\\hline
T7 & Easy to Use                            & 0.5 & 5\\\hline
T8 & Portability                            & 2 & 5\\\hline
T9 & Easy to Change                         & 1 & 3\\\hline
T10 & Concurrency                            & 1 & 0\\\hline
T11& Special Security Features              & 1 & 0\\\hline
T12 & Provides Direct Access for Third Parties & 1 & 0\\\hline
T13 & Special User Training Facilities are Required & 1 & 0\\\hline
\end{tabularx}
\caption{Technical Complexity Factors (TCF) and their Perceived Impact Factors}
\end{table}

\vspace{8pt}
Perceived Impact Factor:
\begin{enumerate}
    \item 0: No influence
    \item 3: Average influence
    \item 5: Strong influence
\end{enumerate}

\begin{align}
    TCF     &= 0.6 + (0.01 * (2*0 + 1*3 + 1*3 + 1*3 +1*3 + 0.5*5 + 0.5*5 + \\
            &\phantom{{}=1}2*5+ 1*3 + 1*0+ 1*0 + 1*0 + 1*0))  \notag\\
            &= 0.6 + 0.01 * 30 \\        
            &= 0.9
\end{align}

\item {Calculating Environmental Complexity Factor (ECF)}

\begin{table}[!thb]
\begin{tabularx}{\textwidth}{|l|X|l|l|}
\hline
\textbf{ECF Type} & \textbf{Description} & \textbf{Weight} & \textbf{Perceived Impact Factor} \\
\hline
E1 & Familiarity with the use case domain                 & 1.5 & 3\\\hline
E2 & Part-time workers   & -1 & 3\\\hline
E3 & Analyst Capability & 0.5 & 5\\\hline
E4 & Application Experience                     & 0.5 & 5\\\hline
E5 & Object-oriented experience     & 1 & 5\\\hline
E6 & Motivation & 1 & 5\\\hline
E7 & Difficult Programming Language & -1 & 1\\\hline
E8 & Stable Requirements & 2 & 5\\\hline
\end{tabularx}
\caption{Environmental Complexity Factors (ECF) and their Perceived Impact Factors}
\end{table}

\vspace{8pt}
Perceived Impact Factor (when positive weights):
\begin{enumerate}
    \item 0: No influence
    \item 1: Strong, negative influence
    \item 3: Average influence
    \item 5: Strong, positive influence
\end{enumerate}
Perceived Impact Factor (when negative weights):
\begin{enumerate}
    \item 0: No influence
    \item 1: Strong, favourable influence
    \item 3: Average influence
    \item 5: Strong, unfavourable influence
\end{enumerate}

\begin{align}
    ECF     &= 1.4 + (-0.03 * (1.5*3 + (-1*3) + 0.5*5 + 0.5*5 + 1*5 + \\
            &\phantom{{}=1}1*5 + (-1*1) + 2*5))  \notag\\
            &= 1.4 + (-0.03 * 25.5) \\        
            &= 0.635
\end{align}


\item {Calculating Unadjusted Use Case Points (UUCP)}

\begin{align}
    UCP     &= UUCP * TCF * ECF \\
            &= (UAW + UUCW) * TCF * ECF \\
            &= (3 + 20) * 0.9 * 0.635 \\
            &= 13.1445
\end{align}


\item {Effort Estimation}

\begin{align}
    Effort Estimate     &= UCP * Productivity Factor \\
                        &= 13.1445 * 20 \\
                        &= 262.89 person-hour
\end{align}

\end{enumerate}

\subsection{Effort estimation using Basic COCOMO Model}

*a1 is usually 2.4 and a2 is 1.05 for projects with less than 50 KLOC
Total lines of code = 0.166 KLOC

\begin{align}
    Effort Estimate     &= a1 * (KLOC)^{a2}\\
                        &= 2.4 * 0.166^{1.05} \\
                        &= 0.364 person-months
\end{align}

\subsection{Effort Estimate Difference between using UCP Approach, COCOMO and Actual Effort}
To compare, first we convert the COCOMO model estimation to person-hours. To do this, we suppose each person works 160 hours each month.\\\\
0.364 person-months = 58.24 person-hours\\\\
Comparing this with the effort estimate from the UCP approach (Use Case Points), which is 262.89 person-hours, we can observe a significant difference between the two estimates.\\\\
The Basic COCOMO model (Constructive Cost Model) provides a much lower estimate, potentially due to the simplicity of the model. In the UCP approach, various factors, including technical and environmental factors are assumed for estimation. Additionally, this approach differentiates between actors and use cases. However, in the COCOMO model, only the lines of code are considered irrespective of the logic and complexity of those codes for each use case.\\\\
It's essential to ensure that the same parameters and conditions are considered when comparing different models for project effort estimation.
\\

Real effort: 42 hours\\\\
The difference in effort estimates from UCP and COCOMO compared to the actual effort arises from several factors. UCP's subjectivity in assessing complexities and changing project requirements can lead to inaccuracies. Human judgment in UCP, prone to biases and expertise levels, also affects estimates. External unpredictable abilities and the team's skill and efficiency further contribute to the variance. Additionally, UCP's assumption of a traditional development approach might not fit all projects, especially those using Agile methodologies. Finally, the effectiveness of project management significantly influences the alignment of actual effort with estimates. These elements collectively explain the discrepancies between the estimated and actual efforts.

\pagebreak