\section{Problem 8: Establishing Relation between Logical SLOC and WMC}

\subsection{Scatter Plot}

\begin{table}[!htb]
\centering
\begin{tabular}{|l|l|l|}
\hline
\textbf{Class Name} & \textbf{Logical SLOC} & \textbf{WMC} \\
\hline
Calculator           & 38    & 22\\
\hline
CsvProcessor         & 3     & 2\\
\hline
DataValidator        & 6     & 7\\
\hline
\end{tabular}
\end{table}

\begin{figure}[!htb]
\centering
\begin{tikzpicture}
\begin{axis}[title={Logical SLOC and WMC}, 
    xlabel={WMC}, 
    ylabel={Logical SLOC}, enlargelimits=0.2,     
    xmajorgrids=true,
    ymajorgrids=true,
    grid style=dashed]
    \addplot[
        scatter/classes={a={blue}, b={red}},
        scatter, mark=*,
        scatter src=explicit symbolic,
        nodes near coords*={\Label},
        visualization depends on={value \thisrow{label} \as \Label} %<- added value
    ] table[meta=ma]
{data/lslocwmc.dat};
\end{axis}
\end{tikzpicture}

\caption{Scatter Plot (Logical SLOC and WMC)}
\label{scatterplot}

\end{figure}

\vspace{12pt}
Figure \ref{scatterplot} shows the scatter plot between the data for Logical SLOC and WMC obtained from METRICSTICS. It can be noted that as the WMC increases, the logical SLOC increases, thus it can be said that the logical SLOC is directly proportional to the WMC.

\pagebreak
\subsection{Correlation Coefficient}

Since the values of x's and y's are non-normally distributed as per the scatter plot shown above, the Spearman's Rank Correlation Coefficient (r\textsubscript{s}) can be used to find the correlation coefficient as it is a measure of association for attributes values that are not distributed normally.

\vspace{12pt}
\begin{table}[!htb]
\centering
\begin{tabular}{|c|c|c|c|c|c|}
\hline
\textbf{WMC(x\textsubscript{i})} & \textbf{Rank(x\textsubscript{i})} & \textbf{SLOC(y\textsubscript{i})} & \textbf{Rank(y\textsubscript{i})} & \textbf{d} & \textbf{d\textsuperscript{2}} \\
\hline
  22    & 3     & 38    & 3     & 0     & 0  \\
  2     & 1     & 3     & 1     & 0     & 0  \\
  7     & 2     & 6     & 2     & 0     & 0  \\
\hline
\end{tabular}
\caption{Summary of statistics for calculating the r\textsubscript{s}}
\end{table}

\begin{align}
    r_s     &= 1 - \frac{6\sum_{i=1}^{3}d^2}{3^3 - 3} \\  
            &= 1 - \frac{6 * 0}{3^3 - 3} \\        
            &= 1 - 0 \\
            &= 1
\end{align}

The value of r\textsubscript{s} = 1 indicate a strong positive correlation. 
\pagebreak